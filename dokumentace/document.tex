\documentclass{article}

\usepackage[czech]{babel} 
\usepackage[utf8]{inputenc}
\usepackage[T1]{fontenc}
%\usepackage[IL2]{fontenc} % Different type of font for wierdos
\usepackage{graphicx} % Required for the inclusion of images
\usepackage{natbib} % Required to change bibliography style to APA
\usepackage{amsmath} % Required for some math elements 
\usepackage{amsfonts}
\usepackage{amssymb}
\usepackage{booktabs}
\usepackage{epstopdf}
\usepackage{multicol}
\usepackage{color}
\usepackage{float}
\usepackage[total={15.5cm,23.5cm}, top=2.5cm, left=3cm, includefoot]{geometry}
\usepackage[colorlinks=true,linkcolor = black, urlcolor = black, citecolor = black]{hyperref}
\usepackage{enumitem}
\usepackage{subfig}
\usepackage[toc,page]{appendix}
\usepackage{hyperref} 
\usepackage{matlab-prettifier}
\usepackage{datetime}
\usepackage{mathtools}
\usepackage[thinc]{esdiff}
\newcommand{\myparagraph}[1]{\paragraph{#1}\mbox{}\\}

\begin{document}


\begin{titlepage}

	\centering

	{\scshape\LARGE Západočeská univerzita\par}
	{\scshape\Large Fakulta aplikovaných věd \par}
	{\scshape\Large Katerdra Kybernetiky \par}
	{\begin{center}
			\includegraphics[width=0.7\textwidth]{pic/fav.jpg}
		\end{center}}

	{\huge\bfseries Dokumentace k projektu ITE-YELLOW\par} % TODO: lepsi nazev

	\vspace{2cm}

	{\Large\itshape Martin Hamar, Radek Kaupe, Samuel Kokoška\par}

	\vfill

	\vspace{1cm}

	\hskip -1 cm{KKY/ITE} \hfill {Datum: \today }



\end{titlepage}

\section{Úvod}
Tento soubor slouží jako manuál pro projekt týmu Yellow. Manuál by měl obsahovat veškeré informace o zprovoznění všech částí projektů. V případě jakékoliv nejasnosti kontaktujte některého člena z týmu: \texttt{skokoska@students.zcu.cz}, \texttt{kauperad@students.zcu.cz}, \texttt{hamarm@students.zcu.cz}

\section{Zprovoznění Jednotlivých modulů}
Projekt je rozdělen na několik modulů, které spolu přirozeně spolupracují. Tyto moduly se dají hrubě rozdělit na hardwarovou část a na softwarouvou část.
Hardwarová část zahrnuje krabičku a mikropočítač ESP8266 a jeho nastavení. Softwarovou část zahrnuje nastavení databáze, subscribera a spuštění backendu a webové stránky.

\subsection{Krabička}
\subsubsection{Součástky}
Fyzické provedení projektu se skládá z pěti částí
\begin{itemize}
    \item Samotná deska ESP8266
    \item Přední část krabičky se senzory
    \item Zadní část krabičky
    \item Napájecí kabel
    \item Okénko
\end{itemize}
\subsubsection{Sestavení}
Pro přípravu zařízení, případně jeho opravu po výměně součástky, je třeba provést následující kroky
\begin{enumerate}
    \item Ujistěte se, že žádný kabel není volný, tedy že všechny vedou od senzoru k liště
    \item Napojte větší lištu (datovou) na desku tak, aby drátek na koncovém pinu vedl na pin D0 (v rohu desky na opačné straně od microUSB portu)
    \item Napojte menší lištu (napájecí) na desku tak, aby červený drátek vedl na pin 3v3 (v rohu desky u microUSB portu) a černý na pin GND
    \item Vložte desku do zadní části krabičky tak, aby microUSB port byl na straně díry pro napájení
    \item Složte krabičku
    \item Připojte napájecí kabel
    \item Zapojte napájení do zásuvky
    \item Diody na krabičce se rozblikají, po jejich zhasnutí přístroj měří
\end{enumerate}
\subsubsection{Okénko}
Okénko slouží k ochraně světelného senzoru a tedy i prodloužení jeho životnosti. Je tedy doporučeno ho použít ve velmi prašných prostředích, či kdekoli jinde, kde hrozí mechanické poškození či zašpinění.

Jelikož okénko není perfektně průhledné, snižuje naměřenou hodnotu osvětlení o zhruba 5\%, což je důležité zohlednit při stanovování hranic pro odeslání varování. Jelikož okénko není třeba pro celkovou funkci přístroje, je možné ho nepoužít pro lepší přesnost měření.

Pro přidání okénka stačí zasunout jeho kolíky do připravených direk na přední straně přístroje.
\subsection{ESP8266}
\subsubsection{Funkce}
Kompletní zařízení disponuje funkcemi:
\begin{itemize}
    \item Měření teploty. vlhkosti a osvětlení.
    \item Možnost zastoupení dedikovaného senzoru teploty v případě jeho poruchy méně přesným senzorem vlhkosti.
    \item Posílání hodnot indikujících poruchu senzorů.
    \item Synchronizace času s NTP serverem.
    \item Posílání naměřených dat přes MQTT s Quality Of Service 1 - zpráva je doručena alespoň jednou.
    \item Detekce selhání zaslání zprávy a tedy ztráty spojení.
    \item Znovupřipojení se k Wi-Fi a MQTT brokeru v případě ztráty spojení.
    \item Archivace naměřených dat v případě ztráty spojení a jejich znovuposlání po obnovení spojení.
    \item Poslání tracebacku a samovolný reset v případě chyby v hlavní smyčce.
\end{itemize}
\subsubsection{Setup}
Pro použití na Vašem pracovišti je potřeba změnit v souboru main.py konstanty
\begin{itemize}
    \item NET\textunderscore NAME - název Vaší Wi-Fi sítě
    \item NET\textunderscore PASS - heslo k této síti
    \item NTP\textunderscore HOST - adresa serveru pro synchronizaci času
    \item BROKER\textunderscore IP - IP adresa MQTT brokera
    \item BROKER\textunderscore UNAME - uživatelské jméno pro připojení k brokeru
    \item BROKER\textunderscore PASSWD - heslo pro připojení k brokeru
    \item TOPIC - topic, na který budou zprávy zasílány
\end{itemize}
Volitelně je zde možnost měnit parametry
\begin{itemize}
    \item PERIOD\textunderscore SEC - perioda měření a posílání v sekundách
    \item TIMEOUT - maximální doba čekání na potvrzení přijetí zprávy od brokera v sekundách - po jejím uplynutí začne archivace a pokusy o znovupřipojení
    \item RECON\textunderscore PERIOD - perioda pokusů o znovupřipojení se na Wi-Fi a brokera v milisekundách
    \item doporučuje se změnit/vymazat ze struktury zprávy položku team\textunderscore name
\end{itemize}
Společně s main.py je potřeba nahrát do zařízení následující knihovny, pokud tak již nebylo učiněno:
\begin{itemize}
    \item temp\textunderscore sensor.py
    \item light\textunderscore sensor.py
    \item umqtt.py
\end{itemize}
Dále stačí jen zapojit zařízení do napájení. Pokud je vše v pořádku, tak by po připojení se na brokera měla zhasnout LED na pinu 16 (D0 na fyzickém zařízení) a měly by se začít posílat zprávy na daný topic.

\subsection{Databáze}
Obsah sekce "Databáze".

\subsection{Subscriber}
Obsah sekce "Subscriber".

\subsection{Backend}
Obsah sekce "Backend".

\section{Budoucí vývoj}
V blízké budoucnosti se určité moduly (především na straně backendu a frontendu) rozšíří o implementaci přihlašování, FaceId a větší robustnosti. Tyto nové technologie by ovšem na zprovoznění projektu neměly mít vliv a manuál by tedy měl sloužit v budoucnosti stále stejně.
\end{document}