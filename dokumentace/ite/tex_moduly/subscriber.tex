\subsection{Subscriber a připojení k Aimtec AWS}
Před samotným spuštěním subscribera je třeba vyřešit credentials týkající se připojení na Aimtec. je hlavní tabulka, kde se ukládájí veškerá validní data
\subsubsection{Aimtec}
\label{sssec:aimtec}
Pro zajištění posílání dat na Aimtec upravte \verb|.env| soubor, aby obsahoval \verb|AIMTEC_URL=| a zadejte správnou URL adresu. Poté spusťte \verb|aimtec.py| skript, pro získání \verb|TEAM_UUID|. Ten také zadejte do již zmíněného soubouru.
Nyní by měl \verb|.env| soubor vypadat následovně:

\subsubsection{Subscriber}
\label{sssec:subscriber}
Pro zajištění komunikace s brokerem je třeba opět upravit \verb|.env| soubor, je třeba zadat následující položky:
\begin{verbatim}
    BROKER_IP=IP
    BROKER_PORT=PORT  
    BROKER_UNAME=uzivatelske_jmeno
    BROKER_PASSWD =heslo
    TOPIC=topic
\end{verbatim}
Po provedení úprav v sekcích \ref{ssec:db}, \ref{sssec:aimtec} a \ref{sssec:subscriber} by tedy vaše \verb|.env| složka měla vypadat následovně: 

\begin{verbatim}
    DB_USER=uzivatel
    DB_PASSWORD=heslo
    DB_HOST=host
    DB_NAME=nazev_databaze
    AIMTEC_URL=URL
    TEAM_UUID=UUID
    BROKER_IP=IP
    BROKER_PORT=PORT  
    BROKER_UNAME=uzivatelske_jmeno
    BROKER_PASSWD =heslo
    TOPIC=topic
\end{verbatim}
Pokud ano, nyní můžete spustit skript \verb|subscriber_vm.py|. 
\subsubsection{Funkce samotného subscribera}
Subscriber má několik funckí:
\begin{enumerate}
    \item Validace příchozích dat
    \item Příjímání chyb z ESP8266
    \item Ukládání dat do databáze
    \item Posílání dat na Aimtec AWS
\end{enumerate}
\myparagraph{Validace dat} 
Momentálně subscriber validuje:
\begin{enumerate}
    \item správný formát příchozího JSONu
    \item správný název týmu
    \item smysluplné hodnoty teploty, osvětlení a vlhkosti
    \item zda má zpráva jiný \verb|timestamp|, než ta poslední (kvůli použití QoS 1)
    \item zda \verb|timestamp| není více jak hodinu z budoucnosti
\end{enumerate}
\myparagraph{Příjímání chyb z ESP8266} 
U příjímání chyb je třeba zmínit, že je ukládá do složky \verb|subscriber\err\|, kde názvy souborů jsou \verb|timestamp|y, které se v chybové zprávě vyskytují.

\myparagraph{Ukládaní do databáze}
Do "ostré"\ databáze se ukládájí pouze validní data. Pro validační a testovací účely se většina dat (ve správném formátu) uloží do testovací tabulky.
Je to z důvodu možnosti zjištění případných chyb, ať už na straně ESPčka nebo subscribera. Také lze beztrestně manipulovat s daty, aniž by se změny projevili na webové stránce.
\myparagraph{Posílání dat na Aimtec} 
Funkčnost se dá popsat následovně:
Při spuštění skriptu se pokusí přihlásit na Aimtec AWS. Pokud server nespí, proběhne všechno v pořádku a data se budou posílat na Aimtec každou minutu.
V případě, že AWS nefunguje, skript se bude o login pokoušet každých deset vteřin, dokud se mu to nepodaří. V pozadí vše ostatní bude probíhat tak jak má (především samozřejmě ukládání dat do databáze), ale data se na AWS nepošlou. 
Momentálně nefunguje žádné zpětné posílání dat, takže v případě výpadku/spánku AWS serveru, data jsou ztracena.  