\subsection{Krabička}
\subsubsection{Součástky}
Fyzické provedení projektu se skládá z pěti částí
\begin{itemize}
    \item Samotná deska ESP8266
    \item Přední část krabičky se senzory
    \item Zadní část krabičky
    \item Napájecí kabel
    \item Okénko
\end{itemize}
\subsubsection{Sestavení}
Pro přípravu zařízení, případně jeho opravu po výměně součástky, je třeba provést následující kroky:
\begin{enumerate}
    \item Ujistěte se, že žádný kabel není volný, tedy že všechny vedou od senzoru k liště
    \item Napojte větší lištu (datovou) na desku tak, aby drátek na koncovém pinu vedl na pin D0 (v rohu desky na opačné straně od microUSB portu)
    \item Napojte menší lištu (napájecí) na desku tak, aby červený drátek vedl na pin 3v3 (v rohu desky u microUSB portu) a černý na pin GND
    \item Vložte desku do zadní části krabičky tak, aby microUSB port byl na straně díry pro napájení
    \item Složte krabičku
    \item Připojte napájecí kabel
    \item Zapojte napájení do zásuvky
    \item Diody na krabičce se rozblikají, po jejich zhasnutí přístroj měří
\end{enumerate}
\subsubsection{Okénko}
Okénko slouží k ochraně světelného senzoru a tedy i prodloužení jeho životnosti. Je tedy doporučeno ho použít ve velmi prašných prostředích, či kdekoli jinde, kde hrozí mechanické poškození či zašpinění.

Jelikož okénko není perfektně průhledné, snižuje naměřenou hodnotu osvětlení o zhruba 5\%, což je důležité zohlednit při stanovování hranic pro odeslání varování. Jelikož okénko není třeba pro celkovou funkci přístroje, je možné ho nepoužít pro lepší přesnost měření.

Pro přidání okénka stačí zasunout jeho kolíky do připravených direk na přední straně přístroje.