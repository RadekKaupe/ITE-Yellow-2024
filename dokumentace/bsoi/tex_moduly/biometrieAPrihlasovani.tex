\subsection{Přihlášení a biometrická autentizace}
Při implementaci byly využity zdroje dodané vyučujícími ve formě skriptů \verb|faceid_server.py|,\newline \verb|extract_embeddings.py|a \verb|train.sh|.
\myparagraph{Rozšíření backendu o FaceID}
Ve skriptu \verb|faceid_server.py| se nacházejí dva Tornado handlery, první zajišťující komunikaci s frontendem v rámci posílání a ukládání fotek a druhý v rámci detekci obličeje ve fotce.
Původně byly oba handlery provozovány v rámci jedné URL adresy, ale eventuelně byly rozděleny. Aplikace funguje následovně:
\begin{enumerate}
    \item Nepřihlášený uživatel se může buď přihlásit přes uživatelské jméno a heslo (vstupní stránka), přihlásit se přes FaceID nebo se zaregistrovat.
    \item Po registraci musí uživatel čekat na schválení adminem, jinak se nepřhlásí (ani v případě, že již fotky má uložené v rámci biometické autentizace).
    \item Přihlášený uživatel má přístup k celé aplikaci, včetně možnosti nafotit si své fotky pro FaceID a natrénovat model pro autentizaci obličejů.
\end{enumerate} 

\myparagraph{Uživatelské účty a identity}
Je dostupné celkem 9 uživatelských účtů pro příhlášení, konkrétně pro: 
\begin{itemize}
    \item 3 vyučující (Jan Švec, Vlasta Radová, Martin Bulín)% Jan Švec, Vlasta Radová, Martin Bulín (bulinm) a Tomáš Lebeda (TODO)
    \item všechny 3 členy týmu
    \item 1 admin účet, ke kterému mají přístup všichni členové týmu
\end{itemize}
Biometrických identit je v době psaní tohoto dokumentu 7. Jsou vytvořené pro všechny 3 členy týmu, jedna pro kategorii `unknown' (aby se `nové' tváře nemohly přihlásit, jinak by je systém identifikoval jako validního uživatele) a 3 identity učitelů. 
% aby mohlo dojít k právě zmíněné identity `unknown' - IDK
Kategorie `unknown' má celkem 17 fotek, z toho 3 neobsahují lidský obličej, aby se neuronová síť naučila přiřazovat neplatné fotografie neautorizovanému uživateli.
Každý z členů má 10 až 25 fotek, které si každý sám nafotil.
Od každého vyučujícího jsme dostali 10 - 25 fotek. Každý s různým počtem, aby šlo testovat úspěšnost identifikace.
Rozhodovací práh je nastaven na alespoň 80\% pravděpodobnost identifikace uživatele. Byl určen testováním členy týmu za rozumnou hranici mezi bezpečností a přihlášením.
Je nutné podotknout, že identita `unkown' není vůbec uložená v databázi a nemá možnost se do aplikace přihlásit.
\myparagraph{Funkčnost přihlašování}
V případě, že se uživatel přihlásí, je mu nastaven Json Web Token (JWT). Ten řeší zda je uživatelská session validní či nikoliv.
Pokud jsou uživatelské credentials v pořádku, uživatel je zaregistován a schválený adminem, je mu JWT uložen jako secure cookie.
Všechny další žádosti nyní automaticky kontrolují zda JWT neexpiroval nebo s ním nebylo manipulováno. 
JWT má nastavenou dobu expirace jedné hodiny a obsahuje `signaturu' - tajný klíč, který je znám pouze serveru.  
\myparagraph{Odhlašování}
Pro odhlašování byl vytvořen Tornado handler, který vymaže token obsahující cookie a přesměruje uživatele zpět na přihlašovací stránku.