\section{Přihlášení a biometrická autentizace}
Při implementaci jsem využil učiteli dodaných zdrojů, ve formě skriptů \verb|faceid_server.py|,\newline \verb|extract_embeddings.py|a \verb|train.sh|.
\myparagraph{Rozšíření backendu o FaceID}
V prvním skriptu se nacházejí Tornado Handleři, jeden zajišťující komunikaci s frontendem v rámci posílání a ukládání fotek a druhý v rámci detekci obličeje ve fotce.
Předtím oba dva handleři ???fungovali??? v rámci jedné URL adresy. Já je rozdělil. Aplikace funguje následovně:
\begin{enumerate}
    \item Nepřihlášený uživatel se může buď přihlásit přes uživatelské jméno a heslo (vstupní stránka), přihlásit se přes FaceID nebo se zaregistrovat
    \item Po registraci musí uživatel čekat na schválení adminem, jinak se nepřhlásí (ani v případě, že již fotky má uložené v rámci biometické autentizace)
    \item Přihlášený uživatel má přístup k celé aplikaci, včetně možnosti nafotit si své fotky pro FaceID a natrénovat model pro autentizaci obličejů
\end{enumerate} 

\myparagraph{Uživatelské účty a identity}
Máme celkem 9 uživatelských účtů pro příhlášení, pro: 
\begin{itemize}
    \item 4 vyučující% Jan Švec, Vlasta Radová, Martin Bulín (bulinm) a Tomáš Lebeda (TODO)
    \item 3 členy týmu
    \item 1 admin účet, ke kterému mají přístup všichni členové týmu
\end{itemize}
Identit je 8. Jsou vytvořené pro všechny členy týmu, jeden kategorie `unknown' (aby se `nové' tváře nemohli přihlásit, jiank by je systém identifikoval jako validního uživatele) a 3 identity učitelé, aby mohlo dojít k právě zmíněné identity `unknown'.
Kategorie `unknown' má celkem 17 fotek, z toho 3 neobsahují lidský obličej, aby se neuronová síť naučila přiřazovat neplatné fotografie neautorizovanému uživateli.
Každý z členů má 10 až 25 fotek, které si každý sám nafotil.
Od každého vyučujícího jsme dostali 10 - 25 fotek. Každý s různým počtem, aby šlo testovat úspěšnost identifikace.
Rozhodovací práh je nastaven na alespoň 90\% pravděpodobnost identifikace uživatele. 
Je nutné podotknout, že identita `unkown' není vůbec uložená v databázi a nemá možnost se do aplikace přihlásit.
\myparagraph{Funkčnost přihlašování}
V případě, že se člověk přihlásí, je mu nastaven Json Web Token (JWT). Ten řeší zda je uživatelská session je validní či nikoliv.
Pokud jsou uživatelská credentials v pořádku a uživatel je zaregistován a schválený adminem je mu JWT uložen jako secure cookie.
Všechny další žádosti nyní automaticky kontrolují zda je JWT neexpiroval nebo s ním nebylo manipulovánu. 
JWT má totiž nastavenou donbu expirace jedné hodiny a obsahuje `podepsán' tajným klíčem, který je znám pouze serveru.  
\myparagraph{Odhlašování}
Pro odhlašování byl vytvořen Tornado Handler, který vymaže token obsahující cookie a přesměruje uživale zpět na přihlašovací stránku.