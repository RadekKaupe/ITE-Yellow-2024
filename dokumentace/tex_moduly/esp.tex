\subsection{ESP8266}
\subsubsection{Funkce}
Kompletní zařízení disponuje funkcemi:
\begin{itemize}
    \item Měření teploty. vlhkosti a osvětlení.
    \item Možnost zastoupení dedikovaného senzoru teploty v případě jeho poruchy méně přesným senzorem vlhkosti.
    \item Posílání hodnot indikujících poruchu senzorů.
    \item Synchronizace času s NTP serverem.
    \item Posílání naměřených dat přes MQTT s Quality Of Service 1 - zpráva je doručena alespoň jednou.
    \item Detekce selhání zaslání zprávy a tedy ztráty spojení.
    \item Znovupřipojení se k Wi-Fi a MQTT brokeru v případě ztráty spojení.
    \item Archivace naměřených dat v případě ztráty spojení a jejich znovuposlání po obnovení spojení.
    \item Poslání tracebacku a samovolný reset v případě chyby v hlavní smyčce.
\end{itemize}
\subsubsection{Setup}
Pro použití na Vašem pracovišti je potřeba změnit v souboru main.py konstanty
\begin{itemize}
    \item NET\textunderscore NAME - název Vaší Wi-Fi sítě
    \item NET\textunderscore PASS - heslo k této síti
    \item NTP\textunderscore HOST - adresa serveru pro synchronizaci času
    \item BROKER\textunderscore IP - IP adresa MQTT brokera
    \item BROKER\textunderscore UNAME - uživatelské jméno pro připojení k brokeru
    \item BROKER\textunderscore PASSWD - heslo pro připojení k brokeru
    \item TOPIC - topic, na který budou zprávy zasílány
\end{itemize}
Volitelně je zde možnost měnit parametry
\begin{itemize}
    \item PERIOD\textunderscore SEC - perioda měření a posílání v sekundách
    \item TIMEOUT - maximální doba čekání na potvrzení přijetí zprávy od brokera v sekundách - po jejím uplynutí začne archivace a pokusy o znovupřipojení
    \item RECON\textunderscore PERIOD - perioda pokusů o znovupřipojení se na Wi-Fi a brokera v milisekundách
    \item doporučuje se změnit/vymazat ze struktury zprávy položku team\textunderscore name
\end{itemize}
Společně s main.py je potřeba nahrát do zařízení následující knihovny, pokud tak již nebylo učiněno:
\begin{itemize}
    \item temp\textunderscore sensor.py
    \item light\textunderscore sensor.py
    \item umqtt.py
\end{itemize}
Dále stačí jen zapojit zařízení do napájení. Pokud je vše v pořádku, tak by po připojení se na brokera měla zhasnout LED na pinu 16 (D0 na fyzickém zařízení) a měly by se začít posílat zprávy na daný topic.
\end{document}
