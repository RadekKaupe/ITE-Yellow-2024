\subsection{Subscriber a připojení k Aimtec AWS}
Před samotným spuštěním subscribera je třeba vyřešit credentials týkající se připojení na Aimtec. je hlavní tabulka, kde se ukládájí veškerá validní data
\subsubsection{Aimtec}
\label{sssec:aimtec}
Pro zajištění posílání dat na Aimtec upravte \verb|.env| soubor, aby obsahoval \verb|AIMTEC_URL=| a zadejte správnou URL adresu. Poté spusťte \verb|aimtec.py| skript, pro získání \verb|TEAM_UUID|. Ten také zadejte do již zmíněného soubouru.
Nyní by měl \verb|.env| soubor vypadat následovně:

\subsubsection{Subscriber}
\label{sssec:subscriber}
Pro zajištění komunikace s brokerem je třeba opět upravit \verb|.env| soubor, je třeba zadat následující položky:
\begin{verbatim}
    BROKER_IP=IP
    BROKER_PORT=PORT  
    BROKER_UNAME=uzivatelske_jmeno
    BROKER_PASSWD =heslo
    TOPIC=topic
\end{verbatim}
Po provedení úprav v sekcích \ref{ssec:db}, \ref{sssec:aimtec} a \ref{sssec:subscriber} by tedy vaše \verb|.env| složka měla vypadat následovně: 

\begin{verbatim}
    DB_USER=uzivatel
    DB_PASSWORD=heslo
    DB_HOST=host
    DB_NAME=nazev_databaze
    AIMTEC_URL=URL
    TEAM_UUID=UUID
    BROKER_IP=IP
    BROKER_PORT=PORT  
    BROKER_UNAME=uzivatelske_jmeno
    BROKER_PASSWD =heslo
    TOPIC=topic
\end{verbatim}
Pokud ano, nyní můžete spustit skript \verb|subscriber_vm.py|. %TODO: popiš jak funguje a jak řeší aimtec.